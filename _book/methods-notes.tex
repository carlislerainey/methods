\documentclass[]{book}
\usepackage{lmodern}
\usepackage{amssymb,amsmath}
\usepackage{ifxetex,ifluatex}
\usepackage{fixltx2e} % provides \textsubscript
\ifnum 0\ifxetex 1\fi\ifluatex 1\fi=0 % if pdftex
  \usepackage[T1]{fontenc}
  \usepackage[utf8]{inputenc}
\else % if luatex or xelatex
  \ifxetex
    \usepackage{mathspec}
  \else
    \usepackage{fontspec}
  \fi
  \defaultfontfeatures{Ligatures=TeX,Scale=MatchLowercase}
\fi
% use upquote if available, for straight quotes in verbatim environments
\IfFileExists{upquote.sty}{\usepackage{upquote}}{}
% use microtype if available
\IfFileExists{microtype.sty}{%
\usepackage{microtype}
\UseMicrotypeSet[protrusion]{basicmath} % disable protrusion for tt fonts
}{}
\usepackage[margin=1in]{geometry}
\usepackage{hyperref}
\hypersetup{unicode=true,
            pdftitle={Concepts and Computation: A Few Notes on Research Methods in Political Science},
            pdfauthor={Carlisle Rainey},
            pdfborder={0 0 0},
            breaklinks=true}
\urlstyle{same}  % don't use monospace font for urls
\usepackage{natbib}
\bibliographystyle{apalike}
\usepackage{longtable,booktabs}
\usepackage{graphicx,grffile}
\makeatletter
\def\maxwidth{\ifdim\Gin@nat@width>\linewidth\linewidth\else\Gin@nat@width\fi}
\def\maxheight{\ifdim\Gin@nat@height>\textheight\textheight\else\Gin@nat@height\fi}
\makeatother
% Scale images if necessary, so that they will not overflow the page
% margins by default, and it is still possible to overwrite the defaults
% using explicit options in \includegraphics[width, height, ...]{}
\setkeys{Gin}{width=\maxwidth,height=\maxheight,keepaspectratio}
\IfFileExists{parskip.sty}{%
\usepackage{parskip}
}{% else
\setlength{\parindent}{0pt}
\setlength{\parskip}{6pt plus 2pt minus 1pt}
}
\setlength{\emergencystretch}{3em}  % prevent overfull lines
\providecommand{\tightlist}{%
  \setlength{\itemsep}{0pt}\setlength{\parskip}{0pt}}
\setcounter{secnumdepth}{5}
% Redefines (sub)paragraphs to behave more like sections
\ifx\paragraph\undefined\else
\let\oldparagraph\paragraph
\renewcommand{\paragraph}[1]{\oldparagraph{#1}\mbox{}}
\fi
\ifx\subparagraph\undefined\else
\let\oldsubparagraph\subparagraph
\renewcommand{\subparagraph}[1]{\oldsubparagraph{#1}\mbox{}}
\fi

%%% Use protect on footnotes to avoid problems with footnotes in titles
\let\rmarkdownfootnote\footnote%
\def\footnote{\protect\rmarkdownfootnote}

%%% Change title format to be more compact
\usepackage{titling}

% Create subtitle command for use in maketitle
\newcommand{\subtitle}[1]{
  \posttitle{
    \begin{center}\large#1\end{center}
    }
}

\setlength{\droptitle}{-2em}
  \title{Concepts and Computation: A Few Notes on Research Methods in Political
Science}
  \pretitle{\vspace{\droptitle}\centering\huge}
  \posttitle{\par}
  \author{Carlisle Rainey}
  \preauthor{\centering\large\emph}
  \postauthor{\par}
  \predate{\centering\large\emph}
  \postdate{\par}
  \date{2018-01-08}

\usepackage{booktabs}
\usepackage{booktabs}
\usepackage{longtable}
\usepackage{array}
\usepackage{multirow}
\usepackage[table]{xcolor}
\usepackage{wrapfig}
\usepackage{float}
\usepackage{colortbl}
\usepackage{pdflscape}
\usepackage{tabu}
\usepackage{threeparttable}

\usepackage{amsthm}
\newtheorem{theorem}{Theorem}[chapter]
\newtheorem{lemma}{Lemma}[chapter]
\theoremstyle{definition}
\newtheorem{definition}{Definition}[chapter]
\newtheorem{corollary}{Corollary}[chapter]
\newtheorem{proposition}{Proposition}[chapter]
\theoremstyle{definition}
\newtheorem{example}{Example}[chapter]
\theoremstyle{definition}
\newtheorem{exercise}{Exercise}[chapter]
\theoremstyle{remark}
\newtheorem*{remark}{Remark}
\newtheorem*{solution}{Solution}
\begin{document}
\maketitle

{
\setcounter{tocdepth}{1}
\tableofcontents
}
\chapter{Placeholder}\label{placeholder}

\chapter{Placeholder}\label{placeholder-1}

\chapter{Questions}\label{questions}

\section{Normative Questions}\label{normative-questions}

\section{Descriptive Questions}\label{descriptive-questions}

\section{Causal Questions}\label{causal-questions}

\subsection{Meaning}\label{meaning}

\chapter{Models}\label{models}

\section{The Scientific Method}\label{the-scientific-method}

\subsection{Concepts}\label{concepts}

\subsection{Models}\label{models-1}

\subsection{Measurements and
Comparisons}\label{measurements-and-comparisons}

\section{Building Models}\label{building-models}

\section{Evaluating Models}\label{evaluating-models}

\section{Review Exercises}\label{review-exercises}

\chapter{Proportions}\label{proportions}

We can imagine lots of lists that are not numbers. This lists contain
qualitative variables. We might have a lists that contains the class of
each student in the class. Rather than contain numbers, this list
contains values like ``Freshman,'' ``Sophomore,'' ``Junior,'' and
``Senior.''

Similarly, at the end of the semester, I have a list of grades. The most
important list is Letter Grades which contains values like A, B, C, D,
and F. We might have lists that contain partisanship, ideology, regime
type, and so on.

When we deal with lists that contain not-numbers, we cannot use the
average because we cannot sum the not-numbers.

Instead, we use proportions. We choose a particular category and compute
the proportion of a list that fall into that category. Note that this
``particular category''" might contain several values, such as freshman
or sophomore (underclassmen) or A, B, or C (passing grades). For
example, we say a certain proportion of the class are freshman. Or we
say that a certain proportion of a class received an A, B, or C.

\section{Calculating Proportions}\label{calculating-proportions}

To compute a proportion, simply count the number of entries in the list
that fall in the particular category and divide that by the total number
of entries.

\[\text{proportion in the category} = \dfrac{\text{number of entries that fall into the category}}{\text{total number of entries}}\]

For example, take the

\begin{table}

\caption{\label{tab:unnamed-chunk-1}Class standing and letter grades of 15 hypothetical students.}
\centering
\begin{tabular}[t]{l|l|l|l|l|l|l|l|l|l|l}
\hline
Class & Junior & Senior & Senior & Senior & Sophomore & Freshman & Sophomore & Junior & Junior & Senior\\
\hline
Grade & C & C & D & C & B & B & B & B & C & A\\
\hline
\end{tabular}
\end{table}

\textbf{Example}: Compute the percent of sophomores in the class.

\begin{quote}
First count the number of sophomores in the list. We have 2. Next, count
the number of total students in the class. We have 10. Finally, divide
the number of sophomores by total number of students. We have
\(\frac{2}{10}\).
\end{quote}

\begin{quote}
If we want, we can leave this proportion as is, reduce it to a simpler
form, or convert it to a percent. It's usually easiest to convert the
final proportion to a percent. in this cases, we have 20\%.
\end{quote}

\textbf{Exercises}:

\begin{enumerate}
\def\labelenumi{\arabic{enumi}.}
\tightlist
\item
  Compute the percent of students in the class that are sophomores.
  Compute the percent that are juniors or seniors.
\item
  Compute the percent of students that received an F. Compute the
  percent that received an A or a B.
\item
  Compute the percent of seniors that received an A. Compute the percent
  of freshmen that received an A.
\end{enumerate}

\section{The Mathematics of
Proportions}\label{the-mathematics-of-proportions}

The approach above to compute a proportion works perfectly. However, it
does not clearly connect to the past or future discussion of averages.

If we can connect proportions and averages, then we can apply the same
theory to two concepts, cutting our theory and formulas by half.

To connect proportions and averages, we simply convert the list of
not-numbers into a list of numbers. To create the new list of numbers,
simply assign the number 1 to the entries that fall into the particular
category. Assign zero to the other entries. This creates a list of
numbers that contains only 0s and 1s. I call this a 0-1 list. The
proportion is simply the average of this 0-1 list.

\textbf{Note: We can think of a proportion as an average of a 0-1 list.}

Notice that a proportion is a special type of average: it is an average
of a 0-1 list that you create by replacing the entries that fall into
the categories of interest with 1s and replacing the other entries with
zero.

\textbf{Exercise}: Suppose I'm evaluating the admissions process so I
put together a list containing the gender of the admitted graduate
students. This list is M, M, F, F, M, M, M, M, F, M, and M.

Try two ways to calculating the proportion of women admitted. 1. The
inital way I described: count the number of women in the list and divide
that by the total number of entries. 2. The more mathematical way I
described: write down the implied 0-1 list (replace Woman with 1 and
everything else with 0) and then average the list. Are these two
approaches different or the same? Elaborate?

If we understand proportions as a special type of average, then anything
we say about averages applies to proportions too!

\section{SD of a 0-1 List}\label{sd-of-a-0-1-list}

If we convert a list of qualitative values into a 0-1 list, then we can
compute the SD of the new list. Just like with other lists of numbers,
the SD of a 0-1 list is the r.m.s. of the deviations from average. We
can compute the SD the usual way. But the usual way takes time, and if
we have a 0-1 list, we have a shortcut.

\[\text{SD of a 0-1 list} = \sqrt{\text{ave. of list} \times (1 - \text{ave. of list})}\]

Notice that the average of a 0-1 list determines the SD \emph{exactly}.
If the average of a 0-1 list is 0.25, then the SD is exactly
\(\sqrt{0.2 \times 0.8} = 0.4\). If the average of a 0-1 list is 0.5,
then the SD is exactly \(\sqrt{0.5 \times 0.5} = 0.5\).

Because of this, we do not often discuss the variation when working with
lists of qualitative values--the proportion contains all the information
we need.

However, this shortcut will help us later.

\section{Properties of Proportions}\label{properties-of-proportions}

We can say three things about proportions:

\begin{enumerate}
\def\labelenumi{\arabic{enumi}.}
\tightlist
\item
  They are at least 0 and at most 1. If none of the entries fall into
  the category, then the top (numerator) of the proportion equals 0, and
  therefore the whole proportion equals 0. If all of the entries fall
  into the category, then the top of the proportion (numerator) equals
  the bottom of the proportion (denominator), and thefore the whole
  proportion equals 1.
\item
  The proportion that do not fall into a category is 1 minus the
  proportion that fall into a category.
\item
  It is easier to talk about percents. After we finish all our
  calculations, we usually convert proportions to percents by
  multiplying the proportion by 100\%. It usually round this percent to
  the nearest whole percent.
\end{enumerate}

\section{Review Exercises}\label{review-exercises-1}

\begin{enumerate}
\def\labelenumi{\arabic{enumi}.}
\item
  Table \ref{tab:govtcontroldata} shows the partisan control of each
  branch of the 49 U.S. states (excluding Nebraska) in 2011. Compute the
  proportion of states with a Republican governor. Repeat for house and
  senate. Is the proportion of Democratic governors, houses, and senates
  necessarily \(1 - \text{proportion Republican}\)? Why or why not?
  Which party has most control in the U.S. states? Can you think of any
  ways that control of state governments affect national politics?

  \rowcolors{2}{gray!6}{white}

  \begin{table}

  \caption{\label{tab:govtcontroldata}The partisan control of each branch of the 49 (excluding Nebraska) state governments in 2011.}
  \centering
  \fontsize{9}{11}\selectfont
  \begin{tabular}[t]{c|c|c|c}
  \hiderowcolors
  \hline
  state & governor & house & senate\\
  \hline
  \showrowcolors
  Alabama & Republican & Republican & Republican\\
  \hline
  Alaska & Republican & Republican & Split\\
  \hline
  Arizona & Republican & Republican & Republican\\
  \hline
  Arkansas & Democrat & Democrat & Democrat\\
  \hline
  California & Democrat & Democrat & Democrat\\
  \hline
  Colorado & Democrat & Republican & Democrat\\
  \hline
  Connecticut & Democrat & Democrat & Democrat\\
  \hline
  Delaware & Democrat & Democrat & Democrat\\
  \hline
  Florida & Republican & Republican & Republican\\
  \hline
  Georgia & Republican & Republican & Republican\\
  \hline
  Hawaii & Democrat & Democrat & Democrat\\
  \hline
  Idaho & Republican & Republican & Republican\\
  \hline
  Illinois & Democrat & Democrat & Democrat\\
  \hline
  Indiana & Republican & Republican & Republican\\
  \hline
  Iowa & Republican & Republican & Democrat\\
  \hline
  Kansas & Republican & Republican & Republican\\
  \hline
  Kentucky & Democrat & Democrat & Republican\\
  \hline
  Louisiana & Republican & Republican & Democrat\\
  \hline
  Maine & Republican & Republican & Republican\\
  \hline
  Maryland & Democrat & Democrat & Democrat\\
  \hline
  Massachusetts & Democrat & Democrat & Democrat\\
  \hline
  Michigan & Republican & Republican & Republican\\
  \hline
  Minnesota & Democrat & Republican & Republican\\
  \hline
  Mississippi & Republican & Democrat & Democrat\\
  \hline
  Missouri & Democrat & Republican & Republican\\
  \hline
  Montana & Democrat & Republican & Republican\\
  \hline
  Nevada & Republican & Democrat & Democrat\\
  \hline
  New Hampshire & Democrat & Republican & Republican\\
  \hline
  New Jersey & Republican & Democrat & Democrat\\
  \hline
  New Mexico & Republican & Democrat & Democrat\\
  \hline
  New York & Democrat & Democrat & Republican\\
  \hline
  North Carolina & Democrat & Republican & Republican\\
  \hline
  North Dakota & Republican & Republican & Republican\\
  \hline
  Ohio & Republican & Republican & Republican\\
  \hline
  Oklahoma & Republican & Republican & Republican\\
  \hline
  Oregon & Democrat & Split & Democrat\\
  \hline
  Pennsylvania & Republican & Republican & Republican\\
  \hline
  Rhode Island & Non-Major Party & Democrat & Democrat\\
  \hline
  South Carolina & Republican & Republican & Republican\\
  \hline
  South Dakota & Republican & Republican & Republican\\
  \hline
  Tennessee & Republican & Republican & Republican\\
  \hline
  Texas & Republican & Republican & Republican\\
  \hline
  Utah & Republican & Republican & Republican\\
  \hline
  Vermont & Democrat & Democrat & Democrat\\
  \hline
  Virginia & Republican & Republican & Democrat\\
  \hline
  Washington & Democrat & Democrat & Democrat\\
  \hline
  West Virginia & Democrat & Democrat & Democrat\\
  \hline
  Wisconsin & Republican & Republican & Republican\\
  \hline
  Wyoming & Republican & Republican & Republican\\
  \hline
  \end{tabular}
  \end{table}

  \rowcolors{2}{white}{white}
\item
  Obtain a copy of Gerber, Green, and Larimer's 2008 article ``Social
  Pressure and Voter Turnout: Evidence from a Large-scale Field
  Experiment'' published in the \emph{American Political Science Review}
  (Volume 102, Issue 1, pp.~33-48). Complete the following tasks:

  \begin{enumerate}
  \def\labelenumii{\alph{enumii}.}
  \tightlist
  \item
    Read the introduction, pp.~33-34 though ``Social Norms, The Calculus
    of Voting, and Prior Research.'' What question interests the
    authors? What type of question is it (normative, descriptive,
    causal)? Would you say that the question matters? Why?
  \item
    Read the section ``Experimental Design,'' pp.~36-38 though
    ``Results''. Briefly describe the design of the study. When and
    where was the student conducted? Who was included in or excluded
    from the study? How did the researchers assign the subjects to the
    treatment and control groups?
  \item
    Examine the four mailers reproduced on pp.~43-46. Using your
    intuition about voter psychology and behavior, rank these mailers
    from most effective to least effective. Which mailer do you
    suspective makes the recipient most likely to vote? Least likely? Do
    you suspect any of the mailers have a negative effect (i.e.,
    receiving the mailer makes the recipient less likely to vote than if
    she had received no mailer at all)?
  \item
    Table \ref{tab:socialpressuredata} contains the data set
    \texttt{social-pressure}. Use these data to re-compute the
    percentages that the authors present in their Table 2 on p.~38. Fill
    these in the appropriate colum in Table
    \ref{tab:socialpressuredata}.
  \item
    Estimate the average treatment effect by subtracting the proportion
    that voted in the control group from the proportion that voted in
    each treatment group (i.e., groups that received a mailer). Convert
    these changes in proportions to changes in percentages by
    multiplying by 100\%.\footnote{We must take care when discussing
      changes in percentages. A 10\% increase in 50\%, could mean either
      (a) \(0.5 + 0.1 = 0.6 = 60\%\) or (b)
      \(0.5 + (0.5 \times 0.1) = 0.5 + 0.05 = 0.55 = 55\%\). To make the
      change clear, we usually talk about changes in ``percentage
      points,'' which clearly refers to (a). Thoughout these notes, I
      always describe changes in percents using percentage point changes
      (a).} According to these estimates, what treatment is most
    effective? Least effective? Do any treatments have a negative
    effect? Does the treatment effects match your guesses about the
    rankings?
  \item
    Comment on the ethics of this study? Would you describe this study
    as unethical? Why or why not?
  \end{enumerate}

  \rowcolors{2}{gray!6}{white}

  \begin{table}

  \caption{\label{tab:socialpressuredata}The numbers of experimental subjects in each condition and the number of subjects in each condition that voted in Gerber, Green, and Larimer's (2008) experiment.}
  \centering
  \resizebox{\linewidth}{!}{\begin{tabular}[t]{>{}c>{\centering\arraybackslash}p{10em}>{\centering\arraybackslash}p{10em}>{\centering\arraybackslash}p{10em}>{\centering\arraybackslash}p{10em}}
  \hiderowcolors
  \toprule
  Condition & Number of Subjects in Condition & Number of Subjects in Condition that Voted & Percent of Subject in Condition that Voted & Average Treatment Effect (in Percentage Points)\\
  \midrule
  \showrowcolors
  Control & 191243 & 56730 &  & Not Applicable\\
  Civic Duty & 38218 & 12021 &  & \\
  Hawthorne & 38204 & 12316 &  & \\
  Self & 38218 & 13191 &  & \\
  Neighbors & 38201 & 14438 &  & \\
  \bottomrule
  \end{tabular}}
  \end{table}

  \rowcolors{2}{white}{white}
\end{enumerate}

\bibliography{book.bib,packages.bib}


\end{document}
