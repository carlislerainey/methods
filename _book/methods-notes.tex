\documentclass[]{book}
\usepackage{lmodern}
\usepackage{amssymb,amsmath}
\usepackage{ifxetex,ifluatex}
\usepackage{fixltx2e} % provides \textsubscript
\ifnum 0\ifxetex 1\fi\ifluatex 1\fi=0 % if pdftex
  \usepackage[T1]{fontenc}
  \usepackage[utf8]{inputenc}
\else % if luatex or xelatex
  \ifxetex
    \usepackage{mathspec}
  \else
    \usepackage{fontspec}
  \fi
  \defaultfontfeatures{Ligatures=TeX,Scale=MatchLowercase}
\fi
% use upquote if available, for straight quotes in verbatim environments
\IfFileExists{upquote.sty}{\usepackage{upquote}}{}
% use microtype if available
\IfFileExists{microtype.sty}{%
\usepackage{microtype}
\UseMicrotypeSet[protrusion]{basicmath} % disable protrusion for tt fonts
}{}
\usepackage[margin=1in]{geometry}
\usepackage{hyperref}
\hypersetup{unicode=true,
            pdftitle={Concepts and Computation: A Few Notes on Research Methods in Political Science},
            pdfauthor={Carlisle Rainey},
            pdfborder={0 0 0},
            breaklinks=true}
\urlstyle{same}  % don't use monospace font for urls
\usepackage{natbib}
\bibliographystyle{apalike}
\usepackage{longtable,booktabs}
\usepackage{graphicx,grffile}
\makeatletter
\def\maxwidth{\ifdim\Gin@nat@width>\linewidth\linewidth\else\Gin@nat@width\fi}
\def\maxheight{\ifdim\Gin@nat@height>\textheight\textheight\else\Gin@nat@height\fi}
\makeatother
% Scale images if necessary, so that they will not overflow the page
% margins by default, and it is still possible to overwrite the defaults
% using explicit options in \includegraphics[width, height, ...]{}
\setkeys{Gin}{width=\maxwidth,height=\maxheight,keepaspectratio}
\IfFileExists{parskip.sty}{%
\usepackage{parskip}
}{% else
\setlength{\parindent}{0pt}
\setlength{\parskip}{6pt plus 2pt minus 1pt}
}
\setlength{\emergencystretch}{3em}  % prevent overfull lines
\providecommand{\tightlist}{%
  \setlength{\itemsep}{0pt}\setlength{\parskip}{0pt}}
\setcounter{secnumdepth}{5}
% Redefines (sub)paragraphs to behave more like sections
\ifx\paragraph\undefined\else
\let\oldparagraph\paragraph
\renewcommand{\paragraph}[1]{\oldparagraph{#1}\mbox{}}
\fi
\ifx\subparagraph\undefined\else
\let\oldsubparagraph\subparagraph
\renewcommand{\subparagraph}[1]{\oldsubparagraph{#1}\mbox{}}
\fi

%%% Use protect on footnotes to avoid problems with footnotes in titles
\let\rmarkdownfootnote\footnote%
\def\footnote{\protect\rmarkdownfootnote}

%%% Change title format to be more compact
\usepackage{titling}

% Create subtitle command for use in maketitle
\newcommand{\subtitle}[1]{
  \posttitle{
    \begin{center}\large#1\end{center}
    }
}

\setlength{\droptitle}{-2em}
  \title{Concepts and Computation: A Few Notes on Research Methods in Political
Science}
  \pretitle{\vspace{\droptitle}\centering\huge}
  \posttitle{\par}
  \author{Carlisle Rainey}
  \preauthor{\centering\large\emph}
  \postauthor{\par}
  \predate{\centering\large\emph}
  \postdate{\par}
  \date{2018-01-08}

\usepackage{booktabs}
\usepackage{booktabs}
\usepackage{longtable}
\usepackage{array}
\usepackage{multirow}
\usepackage[table]{xcolor}
\usepackage{wrapfig}
\usepackage{float}
\usepackage{colortbl}
\usepackage{pdflscape}
\usepackage{tabu}
\usepackage{threeparttable}

\usepackage{amsthm}
\newtheorem{theorem}{Theorem}[chapter]
\newtheorem{lemma}{Lemma}[chapter]
\theoremstyle{definition}
\newtheorem{definition}{Definition}[chapter]
\newtheorem{corollary}{Corollary}[chapter]
\newtheorem{proposition}{Proposition}[chapter]
\theoremstyle{definition}
\newtheorem{example}{Example}[chapter]
\theoremstyle{definition}
\newtheorem{exercise}{Exercise}[chapter]
\theoremstyle{remark}
\newtheorem*{remark}{Remark}
\newtheorem*{solution}{Solution}
\begin{document}
\maketitle

{
\setcounter{tocdepth}{1}
\tableofcontents
}
\chapter{Questions}\label{questions}

\begin{quote}
``The best scientists and explorers have the attributes of kids! They
ask questions and have a sense of wonder. They have curiosity. `Who,
what, where, why, when, and how!' They never stop asking questions, and
I never stop asking questions, just like a five year old.'' ---Sylvia
Earle, marine biologist
\end{quote}

See also a relevant xkcd \href{https://xkcd.com/242/}{comic}.

In political science, we ask a lot of questions about politics, such as
these questions about marriage equality:

\begin{itemize}
\tightlist
\item
  Should gay and lesbian couples have the same right to marry as
  heterosexual couples?
\item
  What percent of the public supports marriage equality for gays and
  lesbians?
\item
  What explains the recent increase in support for marriage equality?
\end{itemize}

Or these questions about income inequality:

\begin{itemize}
\tightlist
\item
  Should the government redistribute wealth?
\item
  Is income inequality higher or lower in the U.S. than France?
\item
  What are the consequences of income inequality?
\end{itemize}

In answering these questions, we might make \emph{claims} about
politics. Claims are just answers to questions. We might make the
following claims about marriage equality:

\begin{itemize}
\tightlist
\item
  Gay and lesbian couples should have the same right to marry as
  heterosexual couples.
\item
  54\% of the public supports marriage equality.
\item
  Court decisions explain the recent increase in support for marriage
  equality.
\end{itemize}

Or we might make these claims about income inequality:

\begin{itemize}
\tightlist
\item
  The government should not redistribute wealth.
\item
  Income inequality is higher in the U.S. than France.
\item
  Income inequality causes a slower growth rate.
\end{itemize}

Political science is all about \emph{asking} and \emph{answering}
questions. But the best approach to answering a question depends on the
type of question.

I break the questions we might ask (or claims we might make) about
politics into three types: normative, descriptive, and causal. Answering
each type question requires a different approach.

\begin{longtable}[]{@{}lllll@{}}
\toprule
\begin{minipage}[b]{0.17\columnwidth}\raggedright\strut
Type\strut
\end{minipage} & \begin{minipage}[b]{0.17\columnwidth}\raggedright\strut
Description\strut
\end{minipage} & \begin{minipage}[b]{0.14\columnwidth}\raggedright\strut
Marriage Example\strut
\end{minipage} & \begin{minipage}[b]{0.21\columnwidth}\raggedright\strut
Inequality Example\strut
\end{minipage} & \begin{minipage}[b]{0.12\columnwidth}\raggedright\strut
Approach\strut
\end{minipage}\tabularnewline
\midrule
\endhead
\begin{minipage}[t]{0.17\columnwidth}\raggedright\strut
normative\strut
\end{minipage} & \begin{minipage}[t]{0.17\columnwidth}\raggedright\strut
How \emph{should} the world look? Asks for a moral judgement.\strut
\end{minipage} & \begin{minipage}[t]{0.14\columnwidth}\raggedright\strut
Should gay and lesbian couples have the same right to marry as
heterosexual couples?\strut
\end{minipage} & \begin{minipage}[t]{0.21\columnwidth}\raggedright\strut
Should the government redistribute wealth?\strut
\end{minipage} & \begin{minipage}[t]{0.12\columnwidth}\raggedright\strut
logic and reasoning\strut
\end{minipage}\tabularnewline
\begin{minipage}[t]{0.17\columnwidth}\raggedright\strut
descriptive\strut
\end{minipage} & \begin{minipage}[t]{0.17\columnwidth}\raggedright\strut
How \emph{does} the world look? Asks for an empirical observation.\strut
\end{minipage} & \begin{minipage}[t]{0.14\columnwidth}\raggedright\strut
What percent of the public supports marriage equality for gays and
lesbians?\strut
\end{minipage} & \begin{minipage}[t]{0.21\columnwidth}\raggedright\strut
Is income inequality higher or lower in the U.S. than France?\strut
\end{minipage} & \begin{minipage}[t]{0.12\columnwidth}\raggedright\strut
observation and measurement\strut
\end{minipage}\tabularnewline
\begin{minipage}[t]{0.17\columnwidth}\raggedright\strut
causal\strut
\end{minipage} & \begin{minipage}[t]{0.17\columnwidth}\raggedright\strut
\emph{Why} does the world look the way it does? What \emph{influences}
X? Asks for a \emph{cause}-and-\emph{effect} relationship or an
\emph{explanation}.\strut
\end{minipage} & \begin{minipage}[t]{0.14\columnwidth}\raggedright\strut
What explains the recent increase in support for marriage
equality?\strut
\end{minipage} & \begin{minipage}[t]{0.21\columnwidth}\raggedright\strut
What are the consequences of income inequality?\strut
\end{minipage} & \begin{minipage}[t]{0.12\columnwidth}\raggedright\strut
observation and measurement, plus clever design\strut
\end{minipage}\tabularnewline
\bottomrule
\end{longtable}

\section{Normative Questions}\label{normative-questions}

Normative questions ask: ``What \emph{should} the world look like?''

In my experience, most people associate political science with normative
questions. When I tell people that I'm a political scientist, they tend
to ask me normative questions.

\begin{enumerate}
\def\labelenumi{\arabic{enumi}.}
\tightlist
\item
  ``You don't think we should invade Iran, do you?'' (Asking for a moral
  judgement about foreign policy.)
\item
  ``What do you think about the breakdown of the family in the U.S?''
  (Implicitly asking for a moral judgement about social policy, i.e.,
  ``Shouldn't the government adopt more pro-family policies?'')
\item
  ``Don't you think we're rewarding laziness?'' (Implicitly asking for a
  moral judgement about economic policy, i.e., ``We shouldn't be doing
  that, should we?''.)
\end{enumerate}

These are normative questions, if perhaps somewhat ill-formed. They are
important questions. Some political scientists, called ``political
philosophers'' or ``normative political theorists,'' focus on these
types of questions.

Some important questions asked by normative political theorists include:

\begin{enumerate}
\def\labelenumi{\arabic{enumi}.}
\tightlist
\item
  Should the state redistribute wealth?
\item
  Under what conditions is war justified?
\item
  What types of behavior should the state regulate?
\item
  How should states make policy?
\end{enumerate}

We will not focus on these types of questions.

However, we all bring normative views with us, and these views are
helpful. Normative views can motivate us to focus on certain descriptive
and causal questions. For example, perhaps you believe that democracy is
the most normatively desirable form of government. This might lead you
to describe how well democracy works in the U.S. (descriptive) or
explain why some countries remain authoritarian (causal). Perhaps you
believe that governments should not torture. This might lead you to
describe the extent to which certain states use torture (descriptive) or
the types of institutional arrangements (independent courts?) that
reduce torture (causal).

Reversing the cycle, answers to descriptive and causal questions might
inform our normative views. For example, if you know that a majority of
the U.S. public supports marriage equality, then you might think the
U.S. should allow gays and lesbians to marry. If you know that income
equality reduces economic growth, then perhaps you think the U.S. should
adopt a more redistributive economic policy.

Normative questions can motivate descriptive and causal questions.
Descriptive and causal questions can inform normative debates. But it is
important to draw a sharp distinction between normative questions and
descriptive/causal questions, because the two require completely
different approaches.

For this class, we'll not focus at all on normative questions. Instead,
we'll focus on descriptive and causal questions.

\section{Descriptive Questions}\label{descriptive-questions}

Descriptive questions ask: ``What \emph{does} the world look like?''

Descriptive questions ask for simple observations---a description of the
world.

For example, we might want to ask the following questions:

\begin{enumerate}
\def\labelenumi{\arabic{enumi}.}
\tightlist
\item
  How many chambers does the Swedish legislature have?
\item
  What percent of voters voted for Barack Obama in 2008?
\item
  How many political parties are there in the United Kingdom?
\item
  What percent of countries today are democracies? How has this changed
  over time?
\item
  What percent of eligible voters actually voted in the U.S. in 2010?
  How does this compare with turnout in other countries?
\item
  What percent of states allow same-sex marriage?
\item
  How polarized is the U.S. Congress? How has this changed over time?
\end{enumerate}

Answering these questions requires some sort of conceptualization (i.e.,
what do we mean by ``polarized''?) and measurement (i.e., how can we
quantify ``polarization''?). But all that is required is observation.
All we need to do make the appropriate measurements (i.e., gather data).

\section{Causal Questions}\label{causal-questions}

Causal questions ask: ``\emph{Why} does the work look the way it does?''

Causal questions ask about a cause or an effect. They ask for an
explanation--\emph{why} did something happen? We might be interested in
the following causal questions:

\begin{enumerate}
\def\labelenumi{\arabic{enumi}.}
\tightlist
\item
  Why is income inequality so high in the U.S.? Why is it growing so
  fast at the moment?
\item
  What causes war between two countries?
\item
  Why do some states become democratic while others remain
  authoritarian?
\item
  What is the effect of an independent court of last resort?
\item
  What explains low turnout in the U.S.?
\item
  Why do some countries have many political parties and other countries
  have few?
\item
  Why does policy change rapidly in some times and/or places, but slowly
  in others?
\item
  Does presidentialism cause democratic failure?
\end{enumerate}

Causal questions and claims are about action. We have one variable
acting on another. There are lots of verbs that summarize action:
causes, influences, affects, changes, increases, decreases, etc. Causal
questions ask us to use these sorts of verbs to describe the way the
world works.

\subsection{Meaning}\label{meaning}

We use the word ``cause'' quite a bit in everyday language. We might
say, for example, that smoking \emph{causes} cancer. In making causal
claims about politics, we might say things like ``wealth \emph{causes}
democracy'' or ``education \emph{causes} turnout.''

But what do these causal claims really mean? What does it mean for
something to \emph{cause} something else?

The idea of causation relies on the counterfactual. The counterfactual
requires us to imagine a world that does not exist (i.e., runs counter
to fact).

For example, suppose Amy has a college degree and voted in 2012. But we
want to know if the college degree caused Amy to vote. In order to
answer that question, we simply need to consider the counterfactual
world in which Amy did not receive a college degree.

We might imagine rewinding time and simply removing Amy's opportunity to
attend college (but nothing else), then letting time move forward to
2012 and observing whether Amy votes. If Amy does not vote in the
counterfactual world, then we say that the college degree caused Amy to
vote. If Amy does vote in the counterfactual world, then we say that the
college degree did not cause Amy to vote.

\begin{figure}
\centering
\includegraphics{diagrams/cf-amy.png}
\caption{Effect of education on turning out to vote.}
\end{figure}

\chapter{Models}\label{models}

\section{The Scientific Method}\label{the-scientific-method}

Most of us learned the scientific method as a rote process, something
like the following:

\begin{itemize}
\tightlist
\item
  Question
\item
  Hypothesis
\item
  Experiment
\item
  Analysis
\item
  Conclusion
\end{itemize}

But I don't think that's how any science, especially social science,
really works. Science is imaginative. It's creative. It's much more like
abstract painting or song-writing or poetry than replacing books in the
library (the coldest, most mechanical task I can think of).

So if science isn't rote hypothesizing and experimenting, what is it?

Let's look at Albert Einstein. Here are some things he said about his
approach to science:

\begin{quote}
When I examine myself and my methods of thought, I come close to the
conclusion that the gift of imagination has meant more to me than any
talent for absorbing absolute knowledge.
\end{quote}

\begin{quote}
All great achievements of science must start from intuitive knowledge. I
believe in intuition and inspiration\ldots{}. At times I feel certain I
am right while not knowing the reason.
\end{quote}

\begin{quote}
Imagination is more important than knowledge.
\end{quote}

\begin{quote}
I have no special talent. I am only passionately curious.
\end{quote}

In a 1961, the influential political scientist wrote the following:

\begin{quote}
\ldots{}I should like to suggest that empirical political science had
better find a place for speculation. It is a grave though easy error for
students of politics impressed by the achievements of the natural
sciences to imitate all of their methods save the most critical one: the
use of the imagination\ldots{} surely it is imagination that has
generally marked the intelligence of the great scientist, and
speculation---often-times foolish speculation, it turned out later--has
generally preceded great advances in scientific theory.
\end{quote}

The scientific method is (sometimes serendipitous) interaction between
speculation and observation.

\begin{figure}
\centering
\includegraphics{diagrams/speculation-observation.pdf}
\caption{A figure illustrating the relationship importance of
speculation and imagination in science.}
\end{figure}

My version of the scientific method is

\begin{itemize}
\tightlist
\item
  Concepts
\item
  Models
\item
  Measurements
\item
  Comparisons
\end{itemize}

The first two components, concepts and models, we might refer to as
``speculation.'' I like this term, because it emphasizing the carefree,
creative nature scientific notions. Many students are too afraid of
being wrong. ``speculation'' frees us up to use our imagination.

\subsection{Concepts}\label{concepts}

Concepts are simply mental constructs. For our purposes, concepts are
words that we use to describe political entities. For example, we can
use the concept of ``democracy'' and describe the United States as a
democracy. We can use the concept of ``liberal'' and refer to Nancy
Pelosi as a liberal.

Concepts are important because they allow us to communicate and reason
precisely and accurately about political events. In general, ambiguous
concepts need work. We need precise, well-defined concepts. They will
never be perfect, but we should always push ourselves toward more
precision and clarity.

Some concepts are easier to define precisely than others, for example.
``Vote share''" is easy to define. ``Democracy'' and ``ideology'' are
not so easy.

To illustrate, answer the following question: What is a democracy? This
is harder than it seems.

Concepts in political science tend to be more abstract and ill-defined.
But it is essential to define our concepts as precisely as possible.
Before we can answer questions about our concepts, we must know what our
concepts mean!

Does negativity cause an increase or a decrease in turnout? To answer
this, we certainly need to know if ``The incumbent voted for Obamacare''
is considered negativity.

\subsection{Models}\label{models-1}

A model is simply a tentative explanation of observed phenomena, used to
better understand the world. Models need not be accurate in every
respect. A model is sometimes referred to as a theory, explanation, or
story.

Models connect concepts in a causal fashion. One minimal model, for
example, is that increasing the level of democracy in a country leads to
more economic growth. In this case, we have two concepts, democracy and
economic growth, connected causally. Of course, we'd want to elaborate a
bit and flesh this model out into a fuller explanation. Why is it,
exactly, that democracy causes growth?

We can also think of models as logical defenses or justifications of
causal claims. For example, we might take two or three basic principles
or axioms as given, and deduce causal claims from this. For example, we
might assume that people behave in a way that benefits them most
financially. We might also assume that the Republicans tend to adopt
lower tax rates than Democrats. Therefore, it seems reasonable to
conclude that an increase in one's wealth makes them more likely to vote
Republican.

\subsection{Measurements and
Comparisons}\label{measurements-and-comparisons}

We'll spend some time later in the semester talking about measurements
and comparisons. But for now, simply note that measurements are simply
quantifications of our concepts. In order to study a phenomenon
systematically, we often want to assign numbers to the entities of
interest. If we are interested in the concept of economic growth, for
example, then the percent change in GDP is a good way to quantify that
concept. Similarly, if we are interested in democracy, then we might
develop a process for assigning each country a score from -10 to 10,
where -10 represents the least democratic countries and 10 represents
the most democratic countries.

Once we have these measures, we'll want to make the appropriate
comparisons. For example, we might want to compare the average economic
growth in democracies and non-democracies.

\section{Building Models}\label{building-models}

But how do you build a model? How do you put together a compelling
logical defense of a causal claim?

A model of the model-building process:

\begin{itemize}
\tightlist
\item
  Step 1: Observe some facts. If these facts are puzzling, even better.
  (e.g., a large percentage of people vote, countries fight
  wars---notice that our models are what make these facts puzzling)
\item
  Step 2: Look at the facts as though they were the end result of some
  unknown process (model). Then speculate about the process that might
  have produced such a result. We're thinking in terms of causation here
\item
  Step 3: Deduce other results (implications, hypotheses, or
  predictions) from the model.
\item
  Step 4: Then ask yourself whether these other implications are true
  and produce new models if necessary.
\end{itemize}

Some rules of thumb for model-building:

\begin{itemize}
\tightlist
\item
  Rule 1: Think ``process.'' In particular, think about causality. One
  thing leads to another, which causes two changes, which both affect
  the concept we care about.
\item
  Rule 2: Develop interesting implications. An interesting implication
  might be one that would otherwise (i.e., in the absence of the model)
  be counterintuitive. It might also be an implication for which we have
  the appropriate data.
\item
  Rule 3: Look for generality. If you start with a theory about US voter
  behavior, can you generalize it to voting behavior in other places?
  Finding generalizations usually involves generalizing nouns.
\item
  Rule 4: Realize that model-building is a slow process
\item
  Rule 5: Talk about your ideas.
\end{itemize}

\section{Evaluating Models}\label{evaluating-models}

Truth

\begin{itemize}
\tightlist
\item
  Are the implications of the model correct?

  \begin{itemize}
  \tightlist
  \item
    What about the assumptions? In practice, we don't worry much about
    the assumptions for a few reasons. Assumptions are usually
    simplifications (e.g., politicians are office-seeking). Many good
    models are based on simple, but incorrect assumptions.

    \begin{itemize}
    \tightlist
    \item
      Assumptions sometimes cannot be observed directly.
    \item
      Testing assumptions distracts your attention from the implications
      of the model. Get it the habit of exploring and evaluating the
      range of implications of the model.
    \item
      That said, all else equal, we would prefer a model based on
      correct assumptions.
    \end{itemize}
  \end{itemize}
\item
  In particular, the model should allow us to make predictions (note
  that we can have a predictive model, that is not necessarily causal.
  However, causal models should be predictive of future observations.)
\end{itemize}

Question: How could we evaluate the truth of our model of ACA opinions?

\begin{itemize}
\tightlist
\item
  Beware of circular or tautological models (e.g., people do what is in
  their self interest)
\item
  Find critical experiments. The ideal approach is to compare
  alternative models. When we have two competing models, we'd like to
  find a situation in which the two model produce different
  implications. This is a powerful situation, because only one of the
  models can be correct.
\item
  To the extent possible, always think in terms of alternative models,
  as opposed to a single model being right or wrong.
\end{itemize}

Beauty (e.g., Downs)

\begin{itemize}
\tightlist
\item
  Simplicity (office-seeking) - some assert that simpler models are more
  likely to be correct.
\item
  Fertility (produces many implications)
\item
  Surprise (why is turnout so high?)
\end{itemize}

Justice - does the model advance normative goals?

\section{Review Exercises}\label{review-exercises}

\begin{enumerate}
\def\labelenumi{\arabic{enumi}.}
\tightlist
\item
  Describe (my view of) the scientific method. How do concepts, models,
  measurements, and comparisons all fit together.
\item
  Summarize my model of the model-building process.
\item
  What are the five rules of thumb for model-building?
\item
  How do we evaluate models? Which of the three criteria do you feel are
  most and least important?
\item
  What is a critical test and why is it important?
\end{enumerate}

\chapter{Proportions and Percents}\label{proportions-and-percents}

We can imagine lots of lists do not contain numbers. Instead of number,
these lists contain not-numbers. We sometimes refer to not-numbers as
``qualitative'' (as opposed to ``quantitative'') values. We might have a
list that contains the class standing of each student in our class.
Rather than numbers, this list of not-numbers contains qualitative
values like ``Freshman,'' ``Sophomore,'' ``Junior,'' and ``Senior.''

Similarly, at the end of the semester, I have two lists of grades. The
first lists numbers that represent each student's points in the class.
This list of quantitative values ranges from 0 to 100. But you care more
about the second list, which contains each student's letter grade in the
class. This list of not-numbers contains the qualitative values A, B, C,
D, and F.

\textbf{Exercise:} Similarly, we might have lists that contains
partisanships, ideologies, regime types, and so on. What qualitative
values might we find in a list of partisanships? Ideologies? Regime
types?

When we deal with lists that contain qualitative values, \textbf{we
cannot use the average} because we cannot sum the not-numbers.

Instead of an averarge, we use \emph{proportions}. We choose a
particular qualitative values and compute the proportion of the list
that falls has that value. Note that this ``particular category'' might
contain several values, such as freshman or sophomore (underclassmen) or
A, B, or C (passing grades). For example, we say a certain proportion of
the class are freshman. Or we say that a certain proportion of a class
received an A, B, or C.

\section{Calculating Proportions}\label{calculating-proportions}

To calucate a proportion, simply count the number of entries in the list
that fall in the particular category and divide that number by the total
number of entries.

\[\text{proportion in the category} = \dfrac{\text{number of entries that fall into the category}}{\text{total number of entries}}\]

\section{Converting Proportions to
Percents}\label{converting-proportions-to-percents}

Proportions serve an important purpose---they make math easy. However,
the scale makes proportions difficult to write or talk about. To make
the math easy, we use proportions. To make the interpretation easy, we
use percents.

To convert a proportion to a percent, we simply multiply the proportion
times 100\%.

\[\text{percent in the category} = \text{proportion in the category} \times 100\%\].

As a general rule of thumb, I recommend that you convert proportions to
percents, but only \emph{after you complete all calculations.}\footnote{Sometimes,
  calculation that work for proportions also work for percents. However,
  sometimes they do not. Because of these, I recommend doing all
  calculations with proportion and convert the final answer into a
  percent.}

\textbf{Example}: Compute the percent of sophomores in the class from
the data in Table \ref{tab:grades}.

\rowcolors{2}{gray!6}{white}

\begin{table}[!h]

\caption{\label{tab:grades}Class standing and letter grades of 15 hypothetical students.}
\centering
\begin{tabular}[t]{ccc}
\hiderowcolors
\toprule
Name & Class & Grade\\
\midrule
\showrowcolors
Joan & Junior & B\\
John & Senior & B\\
Barbara & Senior & B\\
Eric & Senior & C\\
Sara & Sophomore & A\\
\addlinespace
Frances & Freshman & B\\
Noel & Sophomore & C\\
Robert & Junior & A\\
Joseph & Junior & D\\
Marty & Senior & D\\
\addlinespace
Jason & Freshman & B\\
Benjamin & Freshman & D\\
Amanda & Junior & B\\
Thomas & Freshman & C\\
Estella & Sophomore & B\\
\bottomrule
\end{tabular}
\end{table}

\rowcolors{2}{white}{white}

\begin{quote}
First count the number of sophomores in the list. We have 3. Next, count
the number of total students in the class. We have 15. Finally, divide
the number of sophomores by total number of students. We have
\(\frac{3}{15} = 0.2\).
\end{quote}

\begin{quote}
To convert the proportion 0.2 to a percent, we simply multiply it by
100\%. This gives \(0.2 \times 100\% = 20\%\). We can then say ``20\% of
the students in the class are sophomores.''
\end{quote}

\textbf{Exercises}

\begin{enumerate}
\def\labelenumi{\arabic{enumi}.}
\tightlist
\item
  Compute the percent of students in the class that are sophomores.
  Compute the percent that are juniors or seniors.
\item
  Compute the percent of students that received an F. Compute the
  percent that received an A or a B.
\item
  Compute the percent of seniors that received an A. Compute the percent
  of freshmen that received an A.
\end{enumerate}

\section{The Mathematics of
Proportions}\label{the-mathematics-of-proportions}

The approach I describe above to compute a proportion works perfectly.
However, it does not clearly connect to the past (or future) discussion
of averages.

If we can connect proportions and averages, then we reduce our work
later. By thinking of a proportion as a special kind of average, we can
apply the same ideas to two concepts, cutting the ideas we need to learn
and understand in half.

To see that a proportion is a special type of average, we can simply
convert the list of not-numbers into a list of numbers. To create the
new list of numbers, assign the number 1 to the entries that fall into
the particular category and assign the number 0 to the other entries.
This creates a new list of numbers that contains only 0s and 1s. I call
this type of list a 0-1 list. The proportion is simply the average of
this 0-1 list.

\textbf{Important: We can think of a proportion as an average of a 0-1
list.}

Example: Using a 0-1 list, compute the proportion of students in Table
\ref{tab:grades} that are sophomores.

\begin{quote}
Create a 0-1 list where 1 indicates sophomores and 0 indicates. Table
\ref{tab:gradesw01list} includes this 0-1 list. Now simply average the
0-1 list. We have
\(\frac{\text{sum of 0-1 list}}{\text{number of entries in 0-1 list}} = \frac{3}{15} = 0.2\).
\end{quote}

\rowcolors{2}{gray!6}{white}

\begin{table}[!h]

\caption{\label{tab:gradesw01list}Class standing and letter grades of 15 hypothetical students.}
\centering
\begin{tabular}[t]{cccc}
\hiderowcolors
\toprule
Name & Class & Grade & 0-1 List\textsuperscript{*}\\
\midrule
\showrowcolors
Zachery & Junior & B & 0\\
William & Senior & B & 0\\
Gary & Senior & B & 0\\
Lela & Senior & C & 0\\
Douglas & Sophomore & A & 1\\
\addlinespace
Frances & Freshman & B & 0\\
Taylor & Sophomore & C & 1\\
Justyn & Junior & A & 0\\
Phyllis & Junior & D & 0\\
Joseph & Senior & D & 0\\
\addlinespace
Clayton & Freshman & B & 0\\
Fred & Freshman & D & 0\\
Dot & Junior & B & 0\\
Rudolph & Freshman & C & 0\\
Harry & Sophomore & B & 1\\
\bottomrule
\multicolumn{4}{l}{\textsuperscript{*} 1 indicates sophomores and 0 indicates}\\
\multicolumn{4}{l}{not-sophomores.}\\
\end{tabular}
\end{table}

\rowcolors{2}{white}{white}

\textbf{Exercise:} Suppose I want to evaluate the graduate admissions
process at a university, so I put together a list containing the gender
of the admitted graduate students and obtain M, M, F, F, M, M, M, M, F,
M, and M. Try two ways to calculating the proportion of women admitted.
First, use inital method I described: count the number of women in the
list and divide that by the total number of entries. Next, use the more
mathematical way I described: write down the implied 0-1 list (replace F
with 1 and everything else with 0) and then average the list. Are these
two approaches different or the same? Can you explain why the results
are identical?

\section{SD of a 0-1 List}\label{sd-of-a-0-1-list}

If we convert a list of not-numbers into a 0-1 list, then we can compute
the SD of the new list. Just like with other lists of numbers, the SD of
a 0-1 list is the R.M.S. of the deviations from average. We can compute
the SD the usual way. But, as you well know, the usual way takes time.
If we have a 0-1 list, though, we can use a shortcut.

\[\text{SD of a 0-1 list} = \sqrt{\text{ave. of list} \times (1 - \text{ave. of list})}\]

Notice that the average of a 0-1 list (the proportion) determines the SD
\emph{exactly}. If the average of a 0-1 list is 0.25, then the SD is
exactly \(\sqrt{0.2 \times 0.8} = 0.4\). If the average of a 0-1 list is
0.5, then the SD is exactly \(\sqrt{0.5 \times 0.5} = 0.5\).

Because of this, we usually pay little attention to variation when
working with lists of qualitative values---the proportion (or percent)
contains all the information we need.

However, this shortcut will help us later, so remember it.

Exercise: Suppose a 0-1 list that contains 6 0s and 4 1s. What is
average of this list? What proportion are 1s? What is the SD of the
list?

\section{Two Facts about Proportions and
Percents}\label{two-facts-about-proportions-and-percents}

We can say two things about proportions and percents:

\begin{enumerate}
\def\labelenumi{\arabic{enumi}.}
\tightlist
\item
  A proportion is at least 0 and at most 1. Similarly, the percent is at
  least 0\% and at most 100\%. If none of the entries fall into the
  category, then the top (numerator) of the formula for a proportion
  equals 0, and therefore the whole proportion equals 0. If all of the
  entries fall into the category, then the top of the formula for the
  proportion proportion (numerator) equals the bottom of the proportion
  (denominator), and thefore the whole proportion equals 1.
\item
  The proportion that do not fall into a particular category is 1 minus
  the proportion that fall into that category. It turns out that the
  percent that do not fall into a particular category is 100\% minus the
  percent that fall into that category.
\end{enumerate}

\textbf{Exercises}

\begin{enumerate}
\def\labelenumi{\arabic{enumi}.}
\tightlist
\item
  In a Gallup poll conducted in late December of 2017, 39\% of
  respondents said they ``approve'' of the job that Donald Trump is
  doing as president. What percent did not say they approve? In the same
  survey, 55\% said they ``disapprove'' of the job that Donald Trump is
  doing as president. How is this possible?
\item
  I wrote a computer program to analyze a data set. I found that 120\%
  of citizens living in a district received contact from a political
  campaign. Did I do something wrong?
\end{enumerate}

\section{Review Exercises}\label{review-exercises-1}

\begin{enumerate}
\def\labelenumi{\arabic{enumi}.}
\item
  Table \ref{tab:govtcontroldata} shows the partisan control of each
  branch of the 49 U.S. states (excluding Nebraska) in 2011. Compute the
  proportion of states with a Republican governor. Repeat for house and
  senate. Is the proportion of Democratic governors, houses, and senates
  necessarily \(1 - \text{proportion Republican}\)? Why or why not?
  Which party has most control in the U.S. states? Can you think of any
  ways that control of state governments affect national politics?

  \rowcolors{2}{gray!6}{white}

  \begin{table}[!h]

  \caption{\label{tab:govtcontroldata}The partisan control of each branch of the 49 (excluding Nebraska) state governments in 2011.}
  \centering
  \fontsize{9}{11}\selectfont
  \begin{tabular}[t]{c|c|c|c}
  \hiderowcolors
  \hline
  state & governor & house & senate\\
  \hline
  \showrowcolors
  Alabama & Republican & Republican & Republican\\
  \hline
  Alaska & Republican & Republican & Split\\
  \hline
  Arizona & Republican & Republican & Republican\\
  \hline
  Arkansas & Democrat & Democrat & Democrat\\
  \hline
  California & Democrat & Democrat & Democrat\\
  \hline
  Colorado & Democrat & Republican & Democrat\\
  \hline
  Connecticut & Democrat & Democrat & Democrat\\
  \hline
  Delaware & Democrat & Democrat & Democrat\\
  \hline
  Florida & Republican & Republican & Republican\\
  \hline
  Georgia & Republican & Republican & Republican\\
  \hline
  Hawaii & Democrat & Democrat & Democrat\\
  \hline
  Idaho & Republican & Republican & Republican\\
  \hline
  Illinois & Democrat & Democrat & Democrat\\
  \hline
  Indiana & Republican & Republican & Republican\\
  \hline
  Iowa & Republican & Republican & Democrat\\
  \hline
  Kansas & Republican & Republican & Republican\\
  \hline
  Kentucky & Democrat & Democrat & Republican\\
  \hline
  Louisiana & Republican & Republican & Democrat\\
  \hline
  Maine & Republican & Republican & Republican\\
  \hline
  Maryland & Democrat & Democrat & Democrat\\
  \hline
  Massachusetts & Democrat & Democrat & Democrat\\
  \hline
  Michigan & Republican & Republican & Republican\\
  \hline
  Minnesota & Democrat & Republican & Republican\\
  \hline
  Mississippi & Republican & Democrat & Democrat\\
  \hline
  Missouri & Democrat & Republican & Republican\\
  \hline
  Montana & Democrat & Republican & Republican\\
  \hline
  Nevada & Republican & Democrat & Democrat\\
  \hline
  New Hampshire & Democrat & Republican & Republican\\
  \hline
  New Jersey & Republican & Democrat & Democrat\\
  \hline
  New Mexico & Republican & Democrat & Democrat\\
  \hline
  New York & Democrat & Democrat & Republican\\
  \hline
  North Carolina & Democrat & Republican & Republican\\
  \hline
  North Dakota & Republican & Republican & Republican\\
  \hline
  Ohio & Republican & Republican & Republican\\
  \hline
  Oklahoma & Republican & Republican & Republican\\
  \hline
  Oregon & Democrat & Split & Democrat\\
  \hline
  Pennsylvania & Republican & Republican & Republican\\
  \hline
  Rhode Island & Non-Major Party & Democrat & Democrat\\
  \hline
  South Carolina & Republican & Republican & Republican\\
  \hline
  South Dakota & Republican & Republican & Republican\\
  \hline
  Tennessee & Republican & Republican & Republican\\
  \hline
  Texas & Republican & Republican & Republican\\
  \hline
  Utah & Republican & Republican & Republican\\
  \hline
  Vermont & Democrat & Democrat & Democrat\\
  \hline
  Virginia & Republican & Republican & Democrat\\
  \hline
  Washington & Democrat & Democrat & Democrat\\
  \hline
  West Virginia & Democrat & Democrat & Democrat\\
  \hline
  Wisconsin & Republican & Republican & Republican\\
  \hline
  Wyoming & Republican & Republican & Republican\\
  \hline
  \end{tabular}
  \end{table}

  \rowcolors{2}{white}{white}
\item
  Obtain a copy of Gerber, Green, and Larimer's 2008 article ``Social
  Pressure and Voter Turnout: Evidence from a Large-scale Field
  Experiment'' published in the \emph{American Political Science Review}
  (Volume 102, Issue 1, pp.~33-48). Complete the following tasks:

  \begin{enumerate}
  \def\labelenumii{\alph{enumii}.}
  \tightlist
  \item
    Read the introduction, pp.~33-34 though ``Social Norms, The Calculus
    of Voting, and Prior Research.'' What question interests the
    authors? What type of question is it (normative, descriptive,
    causal)? Would you say that the question matters? Why?
  \item
    Read the section ``Experimental Design,'' pp.~36-38 though
    ``Results''. Briefly describe the design of the study. When and
    where was the student conducted? Who was included in or excluded
    from the study? How did the researchers assign the subjects to the
    treatment and control groups?
  \item
    Examine the four mailers reproduced on pp.~43-46. Using your
    intuition about voter psychology and behavior, rank these mailers
    from most effective to least effective. Which mailer do you
    suspective makes the recipient most likely to vote? Least likely? Do
    you suspect any of the mailers have a negative effect (i.e.,
    receiving the mailer makes the recipient less likely to vote than if
    she had received no mailer at all)?
  \item
    Table \ref{tab:socialpressuredata} contains the data set
    \texttt{social-pressure}. Use these data to re-compute the
    percentages that the authors present in their Table 2 on p.~38. Fill
    these in the appropriate colum in Table
    \ref{tab:socialpressuredata}.
  \item
    Estimate the average treatment effect by subtracting the proportion
    that voted in the control group from the proportion that voted in
    each treatment group (i.e., groups that received a mailer). Convert
    these changes in proportions to changes in percentages by
    multiplying by 100\%.\footnote{We must take care when discussing
      changes in percentages. A 10\% increase in 50\%, could mean either
      (a) \(0.5 + 0.1 = 0.6 = 60\%\) or (b)
      \(0.5 + (0.5 \times 0.1) = 0.5 + 0.05 = 0.55 = 55\%\). To make the
      change clear, we usually talk about changes in ``percentage
      points,'' which clearly refers to (a). Thoughout these notes, I
      always describe changes in percents using percentage point changes
      (a).} According to these estimates, what treatment is most
    effective? Least effective? Do any treatments have a negative
    effect? Does the treatment effects match your guesses about the
    rankings?
  \item
    Comment on the ethics of this study? Would you describe this study
    as unethical? Why or why not?
  \end{enumerate}

  \rowcolors{2}{gray!6}{white}

  \begin{table}[!h]

  \caption{\label{tab:socialpressuredata}The numbers of experimental subjects in each condition and the number of subjects in each condition that voted in Gerber, Green, and Larimer's (2008) experiment.}
  \centering
  \resizebox{\linewidth}{!}{\begin{tabular}[t]{>{}c>{\centering\arraybackslash}p{10em}>{\centering\arraybackslash}p{10em}>{\centering\arraybackslash}p{10em}>{\centering\arraybackslash}p{10em}}
  \hiderowcolors
  \toprule
  Condition & Number of Subjects in Condition & Number of Subjects in Condition that Voted & Percent of Subject in Condition that Voted & Average Treatment Effect (in Percentage Points)\\
  \midrule
  \showrowcolors
  Control & 191243 & 56730 &  & Not Applicable\\
  Civic Duty & 38218 & 12021 &  & \\
  Hawthorne & 38204 & 12316 &  & \\
  Self & 38218 & 13191 &  & \\
  Neighbors & 38201 & 14438 &  & \\
  \bottomrule
  \end{tabular}}
  \end{table}

  \rowcolors{2}{white}{white}
\item
  In an article published in 2013 in the \emph{American Journal of
  Political Science} (Volume 57, Issue 1, pp.~1-14) titled ``Do
  Conditional Cash Transfers Affect Electoral Behavior? Evidence from a
  Randomized Experiment in Mexico,'' Ana L. De La O presents an argument
  that conditional cash transfer programs increase turnout. She
  summarizes her results (p.~11) as follows:

  \begin{quote}
  This article provides evidence on the electoral returns of the Mexican
  CCT by analyzing a unique randomized vari- ation in the duration of
  program benefits across eligible villages. The findings suggest that
  the targeted program led to an increase in voter turnout and incumbent
  vote shares. While previous work focuses on CCT persuasive effects,
  this article shows that the CCT pro-incumbent effects are mainly
  explained by a mobilizing mechanism.
  \end{quote}

  But what is CCT program? De La O summarizes Mexico's policy (p.~3):

  \begin{quote}
  Responding to this context, the administration of President Ernesto
  Zedillo launched Progresa in 1997. The program consists of three
  complementary components: a cash transfer, primarily intended to
  subsidize food expenditure, delivered directly to the female heads of
  poor households; a scholarship, intended to compensate for the
  opportunity cost of child labor, thus enabling children to stay in
  school; and basic health care for all members of the household, with
  particular emphasis on preventive health care. All the components add
  up to an average transfer of US\$35 per month, of which more than 85\%
  is in cash. This is approximately 25\% of the average poor, rural
  household income in the absence of the program. Transfers are paid
  every two months. Progresa is conditional because receipt of benefits
  is contingent upon school attendance, regular medical checkups, and
  attendance at platicas (meetings) related to health, hygiene, and
  nutrition issues.
  \end{quote}

  However, the government rolled out the policy in waves. Some
  locations---an ``early immplementation'' group---received the policy
  by late-1997. Other locations---a ``late implementation''
  group---received the policy in early 2000. Importantly, each location
  was assigned to its group \emph{randomly}.

  De La O was particularly interested in the effect of the policy on
  voter turnout. She writes that ``assignment to early treatment leads
  to a 5 percentage point increase in turnout.'' To see the evidence for
  her claim, Table @ref\{tab:progressatab\} shows the average turnout in
  each condition. This table shows that the average turnout across
  locations in the early condition was 68\% and the average turnout
  across locations in the late condition was 63\%. Based on these
  average turnout rates, it seems like the early implementation of the
  program increased turnout by about 5 percentage points on average.

  \begin{table}[!h]

  \caption{\label{tab:progressatab}The average turnout in each condition in De La O.'s study.}
  \centering
  \begin{tabular}[t]{cc}
  \toprule
  Condition & Average Turnout\\
  \midrule
  Early Implementation & 68\%\\
  Late Implementation & 63\%\\
  \bottomrule
  \end{tabular}
  \end{table}

  For a more detailed look, we can also examine histgrams showing the
  distribution of turnout across locations in the two conditions. Figure
  \ref{fig:progresafig} shows these histograms. Does everything seem in
  order? Or can you spot something wrong? Do you think this likely
  undermines her conclusion, why or why not? (Perhaps you find Table
  @ref\{tab:progresatabmed\} helpful.)

  \begin{center}\includegraphics{methods-notes_files/figure-latex/progressafig-1} \end{center}

  \begin{table}[!h]

  \caption{\label{tab:progressatabmed}The median turnout in each condition in De La O.'s study.}
  \centering
  \begin{tabular}[t]{cc}
  \toprule
  Condition & Median Turnout\\
  \midrule
  Early Implementation & 65\%\\
  Late Implementation & 65\%\\
  \bottomrule
  \end{tabular}
  \end{table}
\end{enumerate}

\bibliography{book.bib,packages.bib}


\end{document}
